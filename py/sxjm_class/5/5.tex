\documentclass{article}
\usepackage[utf8]{inputenc}  % 输入字符集为UTF-8
\usepackage{ctex}  % 支持中文
\usepackage{amsmath}  % 支持数学符号
\begin{document}

\title{圆台型易拉罐的剩余物料计算}
\author{林凯}
\date{}
\maketitle

\section{几何参数定义}

首先,定义圆台的参数:
\begin{itemize}
    \item \( r_1 \): 下底半径
    \item \( r_2 \): 上底半径
    \item \( h \): 圆台的高度
\end{itemize}

\section{圆台体积计算}

圆台的体积 \( V \) 可以通过以下公式计算:
\[
V = \frac{1}{3} \pi h (r_1^2 + r_1 r_2 + r_2^2)
\]

\section{材料总用量}

确定制作一个完整的易拉罐所需的材料体积。假设易拉罐的侧面和底部使用相同的材料,计算出所需材料的体积。

\section{剩余物料计算}

剩余物料可通过以下步骤求解:

\subsection{1. 确定原材料的总量}

计算出所使用的材料的初始体积 \( V_{\text{initial}} \)。

\subsection{2. 减去使用的材料体积}

从初始材料中减去制造易拉罐所需的体积 \( V \)。

\subsection{3. 得出剩余物料}

\[
\text{剩余物料} = V_{\text{initial}} - V
\]

\section{示例}

假设:
\begin{itemize}
    \item 下底半径 \( r_1 = 3 \) cm
    \item 上底半径 \( r_2 = 2 \) cm
    \item 高度 \( h = 10 \) cm
\end{itemize}

1. 计算圆台体积:
\[
V = \frac{1}{3} \pi (10) (3^2 + 3 \times 2 + 2^2) = \frac{1}{3} \pi (10) (9 + 6 + 4) = \frac{1}{3} \pi (10) (19) = \frac{190}{3} \pi \, \text{cm}^3
\]

2. 计算剩余物料,假设初始材料体积为 \( V_{\text{initial}} \)。

最后,通过上述步骤可以得到剩余物料的量。

\end{document}
